\documentclass[10pt]{article}

% amsmath package, useful for mathematical formulas
\usepackage{amsmath}
% amssymb package, useful for mathematical symbols
\usepackage{amssymb}

% cite package, to clean up citations in the main text. Do not remove.
\usepackage{cite}

\usepackage{hyperref}

% line numbers
\usepackage{lineno}

% ligatures disabled
\usepackage{microtype}
\DisableLigatures[f]{encoding = *, family = * }

% rotating package for sideways tables
%\usepackage{rotating}

% If you wish to include algorithms, please use one of the packages below. Also, please see the algorithm section of our LaTeX guidelines (http://www.plosone.org/static/latexGuidelines) for important information about required formatting.
%\usepackage{algorithmic}
%\usepackage{algorithmicx}

% Use doublespacing - comment out for single spacing
%\usepackage{setspace} 
%\doublespacing


% Text layout
\topmargin 0.0cm
\oddsidemargin 0.5cm
\evensidemargin 0.5cm
\textwidth 16cm 
\textheight 21cm

% Bold the 'Figure #' in the caption and separate it with a period
% Captions will be left justified
\usepackage[labelfont=bf,labelsep=period,justification=raggedright]{caption}

% Use the PLoS provided BiBTeX style
\bibliographystyle{plos2009}

% Remove brackets from numbering in List of References
\makeatletter
\renewcommand{\@biblabel}[1]{\quad#1.}
\makeatother


% Leave date blank
\date{}

\pagestyle{myheadings}

\begin{document}


% Title must be 150 characters or less
\begin{flushleft}
{\Large
\textbf{Identifying immune epitopes sites under selection in influenza hemagglutinin}
}
% Insert Author names, affiliations and corresponding author email.
\\
Austin G. Meyer$^{1,2,3, \ast}$, 
Claus O. Wilke$^{1,2}$
\\
\bf{1} Department of Integrative Biology, Institute for Cellular and Molecular Biology, and Center for Computational Biology and Bioinformatics. The University of Texas at Austin, Austin, TX 78712, USA.
\\
\bf{2} Department of Molecular Biosciences, Institute for Cellular and Molecular Biology, The University of Texas at Austin, Austin, TX 78712, USA.
\\
\bf{3} School of Medicine, Texas Tech University Health Sciences Center, Lubbock, TX 79430, USA.
\\
$\ast$ E-mail: austin.meyer@utexas.edu
\end{flushleft}

% Please keep the abstract between 250 and 300 words
\section*{Abstract}
Influenza hemagglutinin is among the most studied proteins in all of viral biology. It is both the most variable gene in flu and the protein most responsible for the seasonal re-infection cycle of the human population. There have been dozens of attempts, utilizing as many different methodologies, to identify the sites that are critical for hemagglutinin's seasonal escape from the host immune system. Most of these techniques use some type of sequence analysis to identify sites that are more variable than one would expect from neutral amino acid substitutions. They often then make the jump to assume highly variable sites are under strong host immune pressure. However, since hemagglutinin is most often analyzed as a test data set for new methodologies in molecular evolution, few investigators try to connect sequence variability to actual immune epitope data. Moreover, in the last decade there has been no attempt to systematically re-analyze flu despite a ten-fold growth in available data and the crystallization of well-established molecular evolutionary techniques. Furthermore, there are a number technical complexities in handling hemagglutinin sequences like ensuring clean sequences and alignments, accurate phylogenies, and unifying site numbering between crystal structures, immature and mature proteins, and DNA sequences. For hemagglutinin H3, we have re-analyzed all currently available sequences and curated all experimental immune epitope data. We find that epitope sites are enriched for sites under positive selection. In addition, we find there are a large number of sites that are under diversifying selection that have no experimental justification for being under immune pressure; likewise there are a large number of epitope sites that are not diversifying selection. 
\section*{Author Summary}



\section*{Introduction}
 

% You may title this section "Methods" or "Models". 
% "Models" is not a valid title for PLoS ONE authors. However, PLoS ONE
% authors may use "Analysis" 
\section*{Materials and Methods}

% Results and Discussion can be combined.
\section*{Results}

% We only support three levels of headings, please do not create a heading level below \subsubsection.
\subsection*{Subsection 1}

\subsubsection*{SubSubsection 1.1}

\subsection*{Subsection 2}

\section*{Discussion}



% Do NOT remove this, even if you are not including acknowledgments.

\section*{Acknowledgments}


\section*{References}

\bibliography{flu_manuscript}

\section*{Figure Legends}
% This section is for figure legends only, do not include
% graphics in your manuscript file.
%
%\begin{figure}
%\caption{
%{\bf Bold the first sentence.}  Rest of figure caption.  
%}
%\label{Figure_label}
%\end{figure}


\section*{Tables}
% 
% See introductory notes if you wish to include sideways tables.
%
% NOTE: Please look over our table guidelines at http://www.plosone.org/static/figureGuidelines#tables to make sure that your tables meet our requirements. Certain types of spacing, cell merging, and other formatting tricks may have unintended results and will be returned for revision.
%
%\begin{table}[!ht]
%\caption{
%\bf{Table title}}
%\begin{tabular}{|c|c|c|}
%table information
%\end{tabular}
%\begin{flushleft}Table caption
%\end{flushleft}
%\label{tab:label}
% \end{table}


\end{document}

